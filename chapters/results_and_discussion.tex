\chapter{Results and Discussion}
\label{chap:results_and_discussion}

This chapter presents the findings of the research...., etc.

\section{Presentation of Results}
\subsection{Statistical Analysis}
Detail the statistical tests conducted, the results obtained, and the statistical significance of the findings. Use tables, figures, and graphs to illustrate the results clearly.
\section{Figures}
An example of image insertion is shown in Figure~\ref{fig1}.
\begin{figure}[h]
    \centering
    \includegraphics[width=0.5\linewidth]{images/Mak-Log.png}
    \caption{Caption for the figure.}
    \label{fig1}
\end{figure}


\section{Subfigures}
An example of sub-figures is shown in Figure~\ref{fig2}. Its variants are shown in Figures~\ref{fig2a}, \ref{fig2b} and \ref{fig2c}.

\begin{figure}[h]
     \centering
     \begin{subfigure}[b]{0.3\textwidth}
         \centering
         \includegraphics[width=\textwidth]{images/Mak-Log.png}
         \caption{First figure.}
         \label{fig2a}
     \end{subfigure}
     \hfill
     \begin{subfigure}[b]{0.3\textwidth}
         \centering
         \includegraphics[width=\textwidth]{images/Mak-Log.png}
         \caption{Next figure.}
         \label{fig2b}
     \end{subfigure}
     \hfill
     \begin{subfigure}[b]{0.3\textwidth}
         \centering
         \includegraphics[width=\textwidth]{images/Mak-Log.png}
         \caption{Another figure.}
         \label{fig2c}
     \end{subfigure}
        \caption{Three subfigures.}
        \label{fig2}
\end{figure}
\section{Discussion}
\section{Limitations of the Study}
Acknowledge the limitations of your study. Discuss how these limitations might affect the generalizability of the results and what future research could do to address these limitations.

\section{Conclusion}
Summarize the key findings and their implications. Briefly preview the next chapter, which will provide a comprehensive conclusion to the thesis.
